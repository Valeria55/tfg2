\thispagestyle{empty}
\begin{center}
\begin{LARGE}
\textbf{Abstract}
\end{LARGE}
\end{center}

\begin{quotation}
    The present study proposes the development of an intelligent purchasing system, utilizing a mathematical model for predicting ingredient demand in a restaurant. The model is based on the use of advanced statistical techniques and forecasting methods to analyze past consumption patterns and accurately predict the quantities of ingredients required in future periods. This will enable the reduction of food waste, optimization of inventory levels, and enhancement of the restaurant's operational efficiency.

    A crucial aspect of this research involves the integration of neural networks into the demand prediction process. Neural networks offer the capability to capture complex relationships between variables and adapt to changes in consumption patterns over time. The implementation of a properly trained neural network will be suggested to refine the predictions derived from the mathematical model, further enhancing the accuracy of demand estimations.
    
    The methodology of this investigation will encompass the collection and analysis of historical sales and purchasing data from the restaurant, as well as the exploration of various forecasting techniques and neural network algorithms. Experiments and comparative tests will be conducted to assess the effectiveness of the proposed system in terms of cost reduction, inventory management improvement, and overall customer satisfaction.
    
    Ultimately, it is anticipated that this thesis will contribute to the development of an intelligent and effective purchasing system that empowers restaurants to make informed and timely decisions regarding ingredient procurement. The combination of a mathematical demand prediction model and the implementation of neural networks represents a novel and promising approach to addressing inventory management challenges in the culinary industry, thereby enhancing operational efficiency and sustainability.
\vspace*{0.5cm}

\noindent {\bf Key words:} 1. Input Demand Forecast
, 2. Matematical Modelling, 3. Artificial Neural network.

\end{quotation}
