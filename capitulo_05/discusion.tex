\fancyhead{}
\fancyfoot{}
\cfoot{\thepage}

\lhead{Discusión}

\chapter{Discusión}

% En este captulo, que tambin suele denominarse ``Conclusiones'', se derivan las conclusiones, se explicitan recomendaciones para otros estudios (por ejemplo, sugerir nuevas preguntas, muestras, instrumentos, lneas de investigacin, etc.) y se indica lo que sigue y lo que debe hacerse. Se analiza la posibilidad de extender los resultados a una poblacin mayor que la del estudio. Se evalan las implicaciones, se establece la manera como se respondieron las preguntas de investigacin, si se cumplieron o no los objetivos, se relacionan los resultados con los estudios existentes (vincular con el marco terico y sealar si los resultados coinciden o no con la literatura previa, en qu s y en qu no). Se reconocen las limitaciones de la investigacin, se destaca la importancia y significado de todo el estudio y la forma como encaja en el conocimiento disponible. Se explican los resultados inesperados y cuando no se verificaron las hiptesis es necesario sealar o al menos especular sobre las razones. Recordar que no se deben repetir aqu los resultados sino que se los debe interpretar. La Discusión debe redactarse de tal manera que se facilite la toma de decisiones respecto de una teora, un curso de accin o una problemtica. Resumiendo, este captulo puede ser conceptualmente y dividido en al menos tres secciones, como se ilustra a continuacin.

\section{Logros alcanzados}
En el transcurso de esta investigación, se logró el desarrollo exitoso de un sistema de compra inteligente basado en el histórico de ventas para mejorar la gestión de compras de una empresa gastronómica en Ciudad del Este. Este logro se basó en la implementación de algoritmos y técnicas de análisis de datos que permitieron predecir la demanda de productos en función del comportamiento de los clientes. A través de una metodología de trabajo estructurada y la colaboración activa con los interesados, se logró cumplir con los objetivos propuestos, generando resultados significativos para la optimización de los procesos de compra y toma de decisiones.
% En el desarrollo de este estudio de predicción de la demanda semanal, se alcanzaron varios hitos sobresalientes que contribuyen significativamente al campo de la gestión de inventarios y al pronóstico de demanda en un contexto de venta de productos perecederos:

% \vspace{1\baselineskip}

% \textbf{Aplicación exitosa de técnicas de aprendizaje automático:} La implementación exitosa de dos técnicas de aprendizaje automático, Regresión Lineal y LSTM, para predecir la demanda de productos a lo largo de una semana representa uno de los mayores logros de este proyecto. Ambos enfoques demostraron su valía al brindar pronósticos precisos y útiles que ayudarán a las empresas a tomar decisiones informadas sobre su inventario.

% \textbf{Análisis detallado de la serie temporal:} El análisis exhaustivo de los datos históricos de ventas permitió identificar patrones y tendencias significativas que afectan la demanda de productos. Este análisis reveló la influencia de múltiples factores, como el mes, el día del mes, el día de la semana, los días festivos y las estaciones, en las ventas. Estos conocimientos son fundamentales para comprender y modelar adecuadamente el comportamiento de la demanda.

% \textbf{Transformación de datos no numéricos:} La conversión de datos no numéricos en variables de entrada numéricas se realizó con éxito, lo que facilitó en gran medida la preparación de los datos para el entrenamiento de los modelos. Este proceso aseguró que todas las características se pudieran utilizar eficazmente en el proceso de modelado.

% \textbf{Elección de Python y Google Colab:} La elección de Python como lenguaje de programación y Google Colab como entorno de desarrollo resultó ser altamente efectiva. Python, con sus numerosas bibliotecas, se ha convertido en el estándar de la industria para la inteligencia artificial y el aprendizaje automático. Google Colab, al ser una plataforma en la nube gratuita con acceso a recursos de GPU, permitió una ejecución eficiente de los modelos y un entorno colaborativo para el equipo de investigación.
% Descripcin de los principales descubrimientos obtenidos como producto de la interpretacin de los resultados de la investigacin.
\section{Solución del problema de investigación}
% La solución del problema de investigación se llevó a cabo siguiendo un proceso bien definido:

% \textbf{Exploración de técnicas de predicción:} Se investigaron múltiples técnicas de predicción, desde modelos estadísticos hasta técnicas de aprendizaje automático. Se descubrió que las técnicas de Machine Learning, particularmente la Regresión Lineal y LSTM, brindaron los resultados más prometedores dadas las complejidades de la serie temporal y la necesidad de capturar patrones no lineales.

% \textbf{Adquisición y análisis de datos:} Se recopilaron datos históricos de ventas de un local gastronómico local que abarcaban un período significativo de tiempo. El análisis de esta serie temporal reveló patrones mensuales, estacionales y diarios que influyen en la demanda de productos.

% \textbf{Conversión de datos no numéricos:} Para utilizar estos datos en el proceso de entrenamiento, se convirtieron características no numéricas, como el mes y el día de la semana, en variables numéricas. Esta transformación fue fundamental para incorporar estas características en los modelos.

% \textbf{Entrenamiento y evaluación de modelos:} Se implementaron dos modelos de predicción, Regresión Lineal y LSTM, y se evaluaron mediante métricas de evaluación, incluido el MAE y el MSE. Los modelos se entrenaron y ajustaron para obtener el mejor rendimiento posible.

% \textbf{Experimentación con hiperparámetros:} Se realizaron experimentos para optimizar los hiperparámetros del modelo LSTM, como la tasa de aprendizaje y la arquitectura de la red. Esto aseguró que el modelo LSTM se ajustara adecuadamente a la serie temporal y produjera predicciones precisas.
% % Aqu se realiza la Discusión propiamente dicha, respondiendo al problema planteado e indicando el nivel de satisfaccin de la solucin lograda.



\begin{itemize}
    \item Los objetivos de esta investigación se han logrado de manera satisfactoria. Hemos desarrollado un sistema de compra inteligente basado en el histórico de ventas, que utiliza algoritmos y técnicas de análisis de datos para predecir la demanda de productos en función del comportamiento de nuestros clientes.

    \item En cuanto al objetivo general, hemos profundizado nuestro conocimiento sobre la lógica de negocio de la empresa gastronómica, identificando los procesos críticos que debían integrarse en el sistema informático. Luego, recopilamos los requisitos del sistema mediante la colaboración estrecha con los usuarios y partes interesadas, lo que garantizó que el sistema abordara sus necesidades de manera efectiva.
    
    \item En relación con los objetivos específicos, hemos modelado la lógica del sistema informático de manera sólida, utilizando técnicas y herramientas apropiadas que aseguran su integridad y eficiencia. Luego, procedimos a codificar el modelo lógico en los lenguajes de programación Python y PHP, lo que nos permitió desarrollar un sistema funcional y adaptable.
    
    \item La depuración de los datos cuantitativos relacionados con el comportamiento del consumidor se realizó minuciosamente, garantizando la calidad de la información utilizada para las predicciones. Finalmente, llevamos a cabo pruebas exhaustivas de usabilidad, accesibilidad, multiplataforma y escalabilidad, lo que nos dio la confianza de que el sistema funciona de manera óptima en diversos contextos.
    
\end{itemize}
\section{Sugerencias para futuras investigaciones}
Si bien este estudio alcanzó logros significativos, existen oportunidades para investigaciones futuras que ampliarían y mejorarían aún más la capacidad de pronóstico de la demanda. Las sugerencias para investigaciones futuras incluyen:

\textbf{Exploración de técnicas avanzadas de aprendizaje profundo:} La aplicación de técnicas de aprendizaje profundo más avanzadas, como redes neuronales recurrentes (RNN) bidireccionales, podría permitir la captura de patrones de demanda a largo plazo y mejorar aún más la precisión del pronóstico.

\textbf{Consideración de datos externos:} La inclusión de datos externos, como condiciones climáticas o promociones específicas, puede mejorar la capacidad de los modelos para adaptarse a eventos externos que afectan la demanda.

\textbf{Desarrollo de una interfaz de usuario:} La creación de una interfaz de usuario amigable o una aplicación práctica permitiría que las empresas ingresen datos en tiempo real y obtengan pronósticos instantáneos, lo que facilitaría la toma de decisiones de inventario.

% \textbf{Investigación sobre interpretabilidad del modelo:} La investigación adicional sobre la interpretabilidad de los modelos puede ayudar a comprender por qué se realizan ciertas predicciones y, por lo tanto, facilitar la toma de decisiones basadas en estas predicciones.

% \textbf{Automatización de búsqueda de hiperparámetros:} La implementación de técnicas de optimización automática de hiperparámetros, como optimización bayesiana o búsqueda de cuadrícula inteligente, puede acelerar y mejorar el proceso de selección de hiperparámetros, lo que permitirá que los modelos se ajusten de manera más eficiente y logren una mayor precisión.

\textbf{Desarrollo de una API de pronóstico:} La creación de una API (Interfaz de Programación de Aplicaciones) permitiría a las empresas integrar fácilmente los modelos de pronóstico de demanda en sus sistemas existentes. Esto facilitaría la obtención de pronósticos en tiempo real y su uso en la toma de decisiones diarias de gestión de inventarios.

% En resumen, este estudio ha avanzado significativamente en la solución del problema de la predicción de la demanda semanal, y se espera que futuras investigaciones sigan ampliando los límites de la precisión y la aplicabilidad de estas predicciones en entornos de ventas minoristas.