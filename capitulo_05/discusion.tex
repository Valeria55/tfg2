\fancyhead{}
\fancyfoot{}
\cfoot{\thepage}

\lhead{Discusión}

\chapter{Discusión}

 Se abordan las cuestiones de investigación planteadas, se muestra la consecución de los objetivos propuestos y, para concluir, se realiza un análisis con el propósito de validar las hipótesis formuladas en el contexto de este sistema de compra inteligente.

\section{Logros alcanzados}
  
\begin{itemize}
    \item  Se logró diseñar e implementar un prototipo funcional del sistema de compra inteligente. Este prototipo permite a los usuarios aprovechar el historial de ventas para tomar decisiones de compra más informadas y eficientes.
    
    \item  Se realizó un exhaustivo análisis de datos para comprender los patrones de demanda  y cómo estos pueden influir en las recomendaciones de compra. Esto incluyó la recopilación, limpieza y procesamiento de datos de demanda reales.
    
    \item Se desarrollaron y evaluaron algoritmos de recomendación personalizados que tienen en cuenta el historial de compras de los usuarios. Estos algoritmos ayudan a sugerir productos relevantes y optimizar las decisiones de compra.
    
    \item El trabajo aporta significativamente al campo de la inteligencia artificial y la toma de decisiones basada en datos al aplicar con éxito técnicas de predicción en el contexto de compras inteligentes.
\end{itemize}

\section{Solución del problema de investigación}

\subsection{General}

\emph{Desarrollar un sistema de compra inteligente basado en histórico de ventas para optimizar la gestión de compras de una empresa gastronómica de Ciudad del Este mediante algoritmos y técnicas de análisis de datos para predecir la demanda de productos en función del comportamiento de los clientes.}

\vspace{1\baselineskip}
Al basarse en datos reales y en el comportamiento de los clientes, el sistema optimiza la gestión de compras al sugerir cantidades y variedades de productos a adquirir. De esta manera, se contribuye a una toma de decisiones más informada y eficiente en la adquisición de insumos gastronómicos, reduciendo costos y minimizando pérdidas por exceso o falta de inventario.

\subsection{Específico}

\begin{enumerate}
\item  \emph{Comprender en profundidad la lógica de negocio de la empresa gastronómica para identificar los procesos críticos que deben ser incluidos en el sistema informático a desarrollar}.

Se llevaron a cabo la investigación bibliográfica sobre el funcionamiento de empresas gastronómicas y visitas presenciales para observar directamente el funcionamiento de la empresa, anotando los procedimientos.


\item  \emph{Recabar los requisitos del sistema con los usuarios y partes interesadas.}

Para recopilar los requisitos del sistema con usuarios y partes interesadas, se utilizaron entrevistas detalladas. Esto permitió comprender sus necesidades y expectativas, y la información recopilada se documentó de manera organizada para definir el sistema con éxito.

\item \emph{Modelar la lógica del sistema informático, utilizando técnicas y herramientas adecuadas, para garantizar su integridad y eficiencia.}

Se emplearon técnicas de modelado y herramientas especializadas para analizar datos históricos de demanda, patrones estacionales y tendencias. Esto permitió la creación de un modelo predictivo que mejora la eficiencia operativa al anticipar la demanda futura, optimizando la gestión de inventario y la asignación de recursos

\item  \emph{Codificar el modelo lógico definido en lenguajes de programación Python y PHP.}

Se tradujo el modelo lógico previamente establecido en lenguajes de programación Python, PHP y otros, permitiendo la creación de algoritmos y funciones específicos para la predicción de la demanda. De esta manera, se garantiza que el sistema informático cuente con un componente crucial para la anticipación y gestión eficiente de la demanda, lo que contribuye a su integridad y eficiencia operativa.

\item \emph{Depurar los datos cuantitativos recopilados sobre el comportamiento del consumidor.}

Se empleó la herramienta SQL para llevar a cabo tareas de limpieza y depuración de los conjuntos de datos, lo que incluyó la eliminación de errores, valores atípicos y datos inconsistentes. Esta depuración aseguró la integridad y calidad de los datos, lo que a su vez facilitó un análisis preciso y confiable. El resultado es una base de datos confiable y precisa que respalda la toma de decisiones informadas y estrategias efectivas.

\item  \emph{Realizar pruebas de usabilidad, accesibilidad, multiplataforma, y escalabilidad.} 

Las pruebas se realizaron siguiendo un proceso estructurado. Para las pruebas de usabilidad, se involucraron usuarios reales o simulaciones, observando su interacción para evaluar la facilidad de uso. Las pruebas de accesibilidad se basaron en pautas de accesibilidad para garantizar el acceso para todos. En cuanto a las pruebas de multiplataforma, se verificó que el sistema funcionara correctamente en diversos dispositivos y navegadores. Las pruebas de escalabilidad implicaron pruebas de carga y rendimiento para evaluar la capacidad del sistema ante volúmenes crecientes de datos y demanda.

\end{enumerate}

\section{Análisis de hipótesis}
El sistema de predicción cumple con la hipótesis de acierto en un 70\% no obtante no garantiza automáticamente una reducción del 10\% en el costo de materia prima. La relación entre la eficacia de la predicción y el ahorro de costos es influenciada por diversos factores, como la naturaleza específica del proceso de compra, la volatilidad del mercado, la precisión del modelo y la capacidad de adaptación del sistema a cambios en las condiciones del mercado.

\section{Sugerencias para futuras investigaciones}
Si bien este estudio alcanzó logros significativos, existen oportunidades para investigaciones futuras que ampliarían y mejorarían aún más la capacidad de pronóstico de la demanda. Las sugerencias para investigaciones futuras incluyen:

\textbf{Exploración de técnicas avanzadas de aprendizaje profundo:} La aplicación de técnicas de aprendizaje profundo más avanzadas, como redes neuronales recurrentes (RNN) bidireccionales, podría permitir la captura de patrones de demanda a largo plazo y mejorar aún más la precisión del pronóstico.

\textbf{Consideración de datos externos:} La inclusión de datos externos, como condiciones climáticas o promociones específicas, puede mejorar la capacidad de los modelos para adaptarse a eventos externos que afectan la demanda.


\textbf{Automatización de búsqueda de hiperparámetros:} La implementación de técnicas de optimización automática de hiperparámetros, como optimización bayesiana o búsqueda de cuadrícula inteligente, puede acelerar y mejorar el proceso de selección de hiperparámetros, lo que permitirá que los modelos se ajusten de manera más eficiente y logren una mayor precisión.

\textbf{Desarrollo de una API de pronóstico:} La creación de una API (Interfaz de Programación de Aplicaciones) permitiría a las empresas integrar fácilmente los modelos de pronóstico de demanda en sus sistemas existentes. Esto facilitaría la obtención de pronósticos en tiempo real y su uso en la toma de decisiones diarias de gestión de inventarios.
