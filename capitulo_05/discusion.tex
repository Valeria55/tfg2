\fancyhead{}
\fancyfoot{}
\cfoot{\thepage}

\lhead{Discusión}

\chapter{Discusión}

% En este captulo, que tambin suele denominarse ``Conclusiones'', se derivan las conclusiones, se explicitan recomendaciones para otros estudios (por ejemplo, sugerir nuevas preguntas, muestras, instrumentos, lneas de investigacin, etc.) y se indica lo que sigue y lo que debe hacerse. Se analiza la posibilidad de extender los resultados a una poblacin mayor que la del estudio. Se evalan las implicaciones, se establece la manera como se respondieron las preguntas de investigacin, si se cumplieron o no los objetivos, se relacionan los resultados con los estudios existentes (vincular con el marco terico y sealar si los resultados coinciden o no con la literatura previa, en qu s y en qu no). Se reconocen las limitaciones de la investigacin, se destaca la importancia y significado de todo el estudio y la forma como encaja en el conocimiento disponible. Se explican los resultados inesperados y cuando no se verificaron las hiptesis es necesario sealar o al menos especular sobre las razones. Recordar que no se deben repetir aqu los resultados sino que se los debe interpretar. La discusin debe redactarse de tal manera que se facilite la toma de decisiones respecto de una teora, un curso de accin o una problemtica. Resumiendo, este captulo puede ser conceptualmente y dividido en al menos tres secciones, como se ilustra a continuacin.

\section{Logros alcanzados}
% Descripcin de los principales descubrimientos obtenidos como producto de la interpretacin de los resultados de la investigacin.
\section{Solución del problema de investigacin}
% Aqu se realiza la discusin propiamente dicha, respondiendo al problema planteado e indicando el nivel de satisfaccin de la solucin lograda.
\section{Sugerencias para futuras investigaciones}
% Todo trabajo de investigacin, genera invariablemente como producto colateral, otras interrogantes que suelen ameritar seguir con la investigacin. Esto es derivado del caracter abierto, \textit{i.e.}, inacabado, del conocimiento cientfico. En esta seccin se acostumbra hacer referencia a posibles seguimientos de la investigacin indicando las interrogantes que conforman nuevos problemas pasibles de ser indagados.   