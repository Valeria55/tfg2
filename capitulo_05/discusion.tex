\fancyhead{}
\fancyfoot{}
\cfoot{\thepage}

\lhead{Discusi�n}

\chapter{Discusi�n}

En este cap�tulo, que tambi�n suele denominarse ``Conclusiones'', se derivan las conclusiones, se explicitan recomendaciones para otros estudios (por ejemplo, sugerir nuevas preguntas, muestras, instrumentos, l�neas de investigaci�n, etc.) y se indica lo que sigue y lo que debe hacerse. Se analiza la posibilidad de extender los resultados a una poblaci�n mayor que la del estudio. Se eval�an las implicaciones, se establece la manera como se respondieron las preguntas de investigaci�n, si se cumplieron o no los objetivos, se relacionan los resultados con los estudios existentes (vincular con el marco te�rico y se�alar si los resultados coinciden o no con la literatura previa, en qu� s� y en qu� no). Se reconocen las limitaciones de la investigaci�n, se destaca la importancia y significado de todo el estudio y la forma como encaja en el conocimiento disponible. Se explican los resultados inesperados y cuando no se verificaron las hip�tesis es necesario se�alar o al menos especular sobre las razones. Recordar que no se deben repetir aqu� los resultados sino que se los debe interpretar. La discusi�n debe redactarse de tal manera que se facilite la toma de decisiones respecto de una teor�a, un curso de acci�n o una problem�tica. Resumiendo, este cap�tulo puede ser conceptualmente y dividido en al menos tres secciones, como se ilustra a continuaci�n.

\section{Logros alcanzados}
Descripci�n de los principales descubrimientos obtenidos como producto de la interpretaci�n de los resultados de la investigaci�n.
\section{Soluci�n del problema de investigaci�n}
Aqu� se realiza la discusi�n propiamente dicha, respondiendo al problema planteado e indicando el nivel de satisfacci�n de la soluci�n lograda.
\section{Sugerencias para futuras investigaciones}
Todo trabajo de investigaci�n, genera invariablemente como producto colateral, otras interrogantes que suelen ameritar seguir con la investigaci�n. Esto es derivado del caracter abierto, \textit{i.e.}, inacabado, del conocimiento cient�fico. En esta secci�n se acostumbra hacer referencia a posibles seguimientos de la investigaci�n indicando las interrogantes que conforman nuevos problemas pasibles de ser indagados.   