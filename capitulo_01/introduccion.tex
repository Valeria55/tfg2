\fancyhead{}
\fancyfoot{}
\lhead{Introducción}
\cfoot{\thepage}

\chapter{Introducción}

En el dinámico entorno de la industria gastronómica, la gestión eficiente de recursos se erige como una prioridad estratégica para alcanzar el éxito operativo y financiero. Entre estos recursos, los insumos desempeñan un papel fundamental, y su manejo adecuado no solo impacta la calidad y la consistencia de los platillos ofrecidos, sino también la rentabilidad y sostenibilidad del negocio en su conjunto. En este contexto, emerge la necesidad de desarrollar soluciones innovadoras y tecnológicas que permitan anticipar y ajustar la demanda de insumos de manera inteligente y precisa.

La gestión de un restaurante, es la actividad de dirección coordinada, relacionada con la oferta y el servicio de alimentos y bebidas en empresas de restauración, cumpliendo los requisitos que exige dicha actividad, con el objetivo de satisfacer necesidades y expectativas de sus clientes.

En el contexto actual de competitividad, es común que las empresas de servicios busquen mejorar sus sistemas de gestión, y adopten métodos y principios gerenciales de empresas industriales para aumentar su productividad y calidad. Sin embargo, es importante tener en cuenta que cada tipo de empresa tiene sus particularidades y características únicas que deben ser consideradas para una gestión eficiente.

En el caso de las empresas de servicios de restauración, es esencial estudiar y analizar las características de su actividad, y las particularidades del servicio que ofrecen, para poder aplicar los métodos y principios gerenciales de manera efectiva. Esto permitirá una gestión óptima de la entidad, mejorando los resultados económicos y asegurando la permanencia del negocio.

Entre las particularidades de la actividad de restauración, se pueden mencionar la importancia de la atención al cliente, la gestión de la calidad de los alimentos y bebidas, el manejo adecuado de los costos y la gestión de inventarios, entre otros aspectos. Estas características deben ser consideradas al momento de aplicar métodos de mejora de la productividad y calidad, y adaptarlos a las necesidades específicas de la empresa de servicios de restauración.


Es cierto que las empresas de restauración constituyen un sistema operativo empresarial que se compone de varios subsistemas interrelacionados que trabajan juntos para lograr los objetivos empresariales. Uno de estos subsistemas es el de compras, que es fundamental para garantizar la eficacia del sistema de aprovisionamiento y, por tanto, para la planificación, producción y servicio.

El subsistema de compras tiene la responsabilidad de satisfacer las necesidades del subsistema de aprovisionamiento, asegurándose de que los abastecedores con los que se firma convenios cumplan con las especificaciones de calidad establecidas por la empresa de restauración. Además, es importante que el subsistema de compras garantice el abastecimiento de los productos necesarios y de alta calidad para la empresa de restauración.

Mediante los sistemas computacionales se logran procesar cada vez más cantidada de información, y realizar cálculos y operaciones, complejas por lo que realizar predicciones con estos sistemas resultan más confiables. Esta precisión puede ser mejorada ampliamente utilizando varia técnicas de predicción.
\section{Motivación}
Mediante la aplicación de técnicas avanzadas de análisis de datos y aprendizaje automático, este sistema proporcionará recomendaciones
precisas y estratégicas, contribuyendo así al avance de la inteligencia empresarial y ofreciendo una herramienta eficiente para la toma de
decisiones informadas y fundamentadas. El objetivo final es mejorar la
eficiencia operativa, reducir costos y alcanzar ventajas competitivas, permitiendo a las empresas adaptarse y prosperar en un entorno comercial dinámico y desafiante

\section{Definición del problema}
 Este sistema debe ser capaz de utilizar el histórico de ventas del local para predecir
la demanda futura de los productos, y en consecuencia, realizar las compras
de manera más eficiente y rentable para la gerencia del local.

Problema de investigación: \textit{Necesidad de optimizar la gestión de compras de materia prima de un
local gastronómico en Ciudad del Este, utilizando sistema inteligente computarizado basado en el histórico de ventas.}

\section{Objetivos}
\subsection{General}

Desarrollar un sistema de compra inteligente basado en histórico de ventas para optimizar la gestión de compras de una empresa gastronómica de Ciudad del Este mediante algoritmos y técnicas de análisis de datos para predecir la demanda de productos en función del comportamiento de los clientes.

\subsection{Específico}

\begin{enumerate}
\item  Comprender en profundidad la lógica de negocio de la empresa gastronómica para identificar los procesos críticos que deben ser incluidos en el sistema informático a desarrollar.
\item  Recabar los requisitos del sistema con los usuarios y partes interesadas.
\item  Modelar la lógica del sistema informático, utilizando técnicas y herramientas adecuadas, para garantizar su integridad y eficiencia.
\item  Codificar el modelo lógico definido en lenguajes de programación Python y PHP.
\item Depurar los datos cuantitativos recopilados sobre el comportamiento del consumidor.
\item  Realizar pruebas de usabilidad, accesibilidad, multiplataforma, y escalabilidad. 

\end{enumerate}
\section{Hipótesis}

La implementación de un sistema inteligente en un local gastronómico de Ciudad del Este, posibilitará una gestión de compras cuya eficiencia posibilitará reducir el costo de materia prima en al menos 10 \% y predecir al menos con 70 \% de acierto, la variedad y cantidad de productos a ser comprados.

\section{Fundamentación}

En el entorno comercial actual, la toma de decisiones basada en información histórica puede ser un factor clave para el éxito de una empresa. Sin embargo, en muchos casos, esta información no se explota adecuadamente y la gestión de datos se convierte en una tarea compleja.

En el caso de una empresa gastronómica, es importante conocer el rendimiento en ventas y controlar los índices que reflejen el desempeño del negocio. Sin embargo, la simple gestión de estos índices no es suficiente, ya que la empresa puede planificar una determinada cantidad de ventas y adquirir una cantidad de materia prima en consecuencia, pero luego no lograr vender todo lo adquirido y mantener un exceso de stock, o incluso tener productos en inventario que no se rotan.

Es por eso que se plantea el desarrollo de una aplicación que permita proyectar las ventas, para tener un presupuesto real de compras que se adapte a la demanda del mercado y evite el exceso de stock o la falta de rotación de inventarios. De esta manera, la empresa podrá tomar decisiones informadas y maximizar su rendimiento.



\section{Impacto de la investigación}

La implementación del sistema inteligente de compra basado en el histórico de ventas permitirá al gerente de la empresa tomar decisiones informadas y estratégicas, lo que se traducirá en un mejor desempeño económico y financiero de la empresa gastronómica.

Además, la investigación realizada para el desarrollo del sistema también tendrá un impacto positivo en otros estudiantes y futuros proyectos, ya que se podrán utilizar los conocimientos y experiencias adquiridos para mejorar otros procesos y aplicaciones en el ámbito empresarial y tecnológico. En consecuencia, la investigación tendrá un impacto significativo tanto en la empresa como en el ámbito académico y profesional.






\section{Organización del trabajo.} 
El presente trabajo está organizado de la siguiente manera:

\vspace{1\baselineskip}
\textbf{Capítulo 1:} 
Este capítulo introductorio presenta la motivación para la investigación, que se enfoca en la gestión eficiente de insumos en la industria gastronómica. El problema de investigación se centra en la optimización de la gestión de compras de materia prima en un restaurante de Ciudad del Este. Se establecen los objetivos, hipótesis y la importancia del estudio, destacando su impacto positivo en la empresa y en futuros proyectos.

\vspace{1\baselineskip}
\textbf{Capítulo 2:} 
Se presentan algunos conceptos relacionados con las predicciones de compras de insumos, cómo funciona la relación de demanda y gestion de suministro de un local gastronómico, por qué es importante la predicción de compra de insumos para el planeamiento futuro tanto a corto, mediano y largo plazo, los factores que influyen en el comportamiento de la curva de insumos demandados y tiempos actuales en las predicciones de demanda. 

\vspace{1\baselineskip}
\textbf{Capítulo 3:} 
En este capítulo se aborda la afinación de hiperparámetros en un modelo de predicción, utilizando el porcentaje de pérdida Root Mean Square (RMS) como métrica guía para evaluar el rendimiento del modelo. Se describen los procedimientos para ajustar diferentes hiperparámetros, como la tasa de aprendizaje y el número de capas ocultas, con el objetivo de minimizar la pérdida RMS y, por lo tanto, mejorar la precisión y eficiencia del modelo de predicción. Se detalla la metodología empleada en este proceso, incluyendo la utilización de conjuntos de validación y estrategias de búsqueda de hiperparámetros, y se presentan los resultados obtenidos tras la afinación.

\vspace{1\baselineskip}
\textbf{Capítulo 4:}
En este capítulo se presentan y analizan los resultados de las predicciones
de compras con los modelos de  predicción estudiados. 

\vspace{1\baselineskip}
\textbf{Capítulo 5:}
En el último capítulo de esta tesis, se procede a la discusión y análisis de los resultados obtenidos. Se lleva a cabo una evaluación de la eficacia del modelo de predicción y se reflexiona sobre su relevancia en el marco del sistema de compra inteligente basado en el historial de ventas. Además, se destacan áreas potenciales para investigaciones futuras en el contexto de este sistema.
