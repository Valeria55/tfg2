\fancyhead{}
\fancyfoot{}
\lhead{Introducción}
\cfoot{\thepage}

\chapter{Introducción}

Este capítulo típicamente realiza la presentación de todo el Trabajo Final de Grado (TFG), excepto por las conclusiones que no deben ser adelantadas aquí. Se considera este capítulo como el inicio de la parte textual del informe del trabajo, toda la redacción preliminar a la introducción corresponde así a la parte pretextual del mismo. Debería incluir, generalmente en este orden \cite{sampieri}.

\section{Motivación}
La motivación que condujo al autor a seleccionar el tema y emprender la investigación. Así, se trata de un contexto dependiente enteramente de los gustos e intereses propios del autor.

La motivación que condujo al autor a seleccionar el tema y emprender la investigación. Así, se trata de un contexto dependiente enteramente de los gustos e intereses propios del autor.

\section{Definición del problema}
Debido al bajo rendimiento que poseen los postulantes en los exámenes de Física durante el curso de admisión a las carreras de ingeniería de la Facultad Politécnica de la Universidad Nacional del Este (FPUNE), se plantea adecuar la plataforma Moodle con la utilización de recursos tecnológicos y herramientas basadas en gamificación que será utilizada durante el proceso de aprendizaje, teniendo en cuenta las competencias claves, los contenidos y los objetivos que se especifican en el programa de estudio.

La gamificación ha tomado relevancia en muchas áreas, en especial en la educación, al combinar la mecánica de los juegos con el contexto educativo para conseguir mejores resultados académicos.

Paralelamente, Moodle se ha consolidado como el entorno virtual de sistema de gestión de aprendizaje más extendido a nivel mundial que se destaca por ser de código abierto. Referenciar
 
Con la incorporación de la gamificación se pretende que las experiencias de enseñanza-aprendizaje se tornen más interesantes e interactivas, otorgando a los postulantes un rol activo al protagonizar su proceso educativo.

\textit{
    Necesidad de adecuar la plataforma virtual de aprendizaje Moodle con la técnica de gamificación aplicada al programa de estudio de Física del curso de admisión a la FPUNE.
}

\section{Objetivos, hipótesis, justificación y delimitación del alcance del tratado.}
Es importante una clara definición de cada uno de estos tópicos para facilitar la comprensión de toda la obra. Esto otorga una visión global del trabajo e indica qué de resultados son buscados con el desarrollo del trabajo







\section{Descripción de los contenidos por capítulo.} 
Usualmente, el capítulo termina anunciando brevemente el contenido de los restantes capítulos.

