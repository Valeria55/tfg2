\thispagestyle{empty}
\begin{center}
\begin{LARGE}
\textbf{Resumen}
\end{LARGE}
\end{center}
\begin{quotation}

    Este trabajo propone el desarrollo de un sistema inteligente de compras para restaurantes, que se basa en un modelo matemático de predicción de la demanda de insumos. Este modelo utiliza técnicas estadísticas avanzadas y métodos de pronóstico para analizar patrones de consumo pasados y prever con precisión las cantidades de insumos necesarias en el futuro. Además, se incorpora una red neuronal entrenada para mejorar aún más las estimaciones de demanda, permitiendo así reducir el desperdicio de alimentos, optimizar los niveles de inventario y mejorar la eficiencia operativa del restaurante. La metodología de investigación involucra la recopilación y análisis de datos históricos de ventas y compras, así como la exploración de diversas técnicas de pronóstico y algoritmos de redes neuronales. Se llevarán a cabo experimentos para evaluar la eficacia del sistema en términos de reducción de costos y gestión de inventario.

    En última instancia, esta tesis busca contribuir al desarrollo de un sistema de compras efectivo que permita a los restaurantes tomar decisiones informadas y oportunas en cuanto a la adquisición de insumos. La combinación de un modelo matemático de predicción de demanda con redes neuronales representa un enfoque innovador para abordar los desafíos de la gestión de inventario en la industria gastronómica, con el objetivo de mejorar la eficiencia y sostenibilidad operativa.
    
    
    
    
    
    
\vspace*{0.5cm}

\noindent {\bf Descriptores:} 1. Predicción de la demanda de insumos, 2. Modelo matemático, 3. Redes neuronales artificiales.

\end{quotation}
