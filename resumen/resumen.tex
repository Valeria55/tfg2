\thispagestyle{empty}
\begin{center}
\begin{LARGE}
\textbf{Resumen}
\end{LARGE}
\end{center}
\begin{quotation}

    El presente trabajo propone el desarrollo de un sistema inteligente de compras, utilizando un modelo matemático de predicción de la demanda de insumos para el restaurante. El modelo se basa en la utilización de técnicas estadísticas avanzadas y métodos de pronóstico para analizar los patrones de consumo pasados y prever con precisión las cantidades de insumos necesarias en períodos futuros. Esto permitirá reducir el desperdicio de alimentos, optimizar los niveles de inventario y mejorar la eficiencia operativa del restaurante.

    Una parte fundamental de este estudio es la incorporación de redes neuronales en el proceso de predicción de la demanda. Las redes neuronales ofrecen la capacidad de capturar relaciones complejas entre variables y adaptarse a los cambios en los patrones de consumo con el tiempo. Se propondrá la implementación de una red neuronal adecuadamente entrenada para refinar las predicciones obtenidas a partir del modelo matemático, lo que aumentará aún más la precisión de las estimaciones de demanda.
    
    La metodología de esta investigación involucrará la recopilación y análisis de datos históricos de ventas y compras del restaurante, así como la exploración de diversas técnicas de pronóstico y algoritmos de redes neuronales. Se llevarán a cabo experimentos y pruebas comparativas para evaluar la eficacia del sistema propuesto en términos de reducción de costos, mejora de la gestión de inventario y satisfacción general del cliente.
    
    En última instancia, se espera que esta tesis contribuya al desarrollo de un sistema de compras inteligente y efectivo que permita a los restaurantes tomar decisiones informadas y oportunas en cuanto a la adquisición de insumos. La combinación de un modelo matemático de predicción de demanda y la implementación de redes neuronales representa un enfoque novedoso y prometedor para abordar los desafíos de la gestión de inventario en la industria gastronómica, mejorando la eficiencia y sostenibilidad operativa.
    
    
    
    
    
    
\vspace*{0.5cm}

\noindent {\bf Descriptores:} 1. Predicción de la demanda de insumos, 2. Modelo matemático, 3. Redes neuronales artificiales.

\end{quotation}
