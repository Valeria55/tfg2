\fancyhead{}
\fancyfoot{}
\newtheorem{teorema}{Teorema}
\cfoot{\thepage}

\lhead{Conceptos fundamentales, teorías y antecedentes}
%\rhead{\today}
%\rfoot{\thepage}

\chapter{Predicciones de Compras}
La predicciónes el proceso de anticipar la cantidad y tipo de insumos que un negocio gastronómico necesitará adquirir en el futuro, con el objetivo de asegurar un suministro adecuado y eficiente.


\section{Relación de Demanda y Gestión de Suministro}

La relación de demanda y gestión de suministro se refiere a la interacción entre la cantidad de productos o insumos que los clientes requieren (demanda) y cómo la empresa se asegura de tener suficientes suministros disponibles para satisfacer esa demanda de manera eficiente. En un negocio gastronómico, la gestión de suministros es crucial para evitar situaciones en las que falten ingredientes clave, lo que podría afectar negativamente la calidad del servicio y la satisfacción del cliente.

\section{Importancia de la Predicción de Compra de Insumos}

La predicción de compra de insumos es vital para la planificación a corto, mediano y largo plazo en un negocio gastronómico por varias razones

\begin{enumerate}
  \item \textbf{Optimización de Inventario:} Al prever la demanda futura, el negocio puede mantener un nivel de inventario óptimo. Comprar en exceso puede llevar al desperdicio de alimentos, mientras que comprar muy poco puede resultar en escasez y pérdida de ventas.
  
  \item \textbf{Reducción de Costos:} Una predicción precisa permite comprar solo lo necesario, lo que reduce los costos asociados con el almacenamiento y la conservación de productos perecederos.
  
  \item \textbf{Eficiencia Operativa:} Saber qué insumos se necesitarán en el futuro permite planificar y programar las operaciones de manera más eficiente, evitando retrasos y problemas logísticos.
  
  \item \textbf{Satisfacción del Cliente:} Mantener un suministro constante de productos esencial para el negocio, ya que los clientes esperan encontrar su elección preferida en el menú en todo momento.
\end{enumerate}

\section{Factores que Influyen en la Curva de Insumos Demandados}

Varios factores pueden influir en el comportamiento de la curva de insumos demandados:

\begin{itemize}
    \item Día de la Semana y Estacionalidad: Los patrones de consumo pueden variar según el día de la semana y la temporada. Por ejemplo, los fines de semana pueden tener una mayor demanda en comparación con los días laborables.
    
    \item Eventos Especiales: Eventos como días festivos, celebraciones locales o conciertos cercanos pueden aumentar la demanda de alimentos y bebidas.
    
    \item Tendencias de Consumo: Las tendencias gastronómicas y las preferencias cambiantes de los consumidores pueden afectar la demanda de ciertos productos.
    
    \item Clima: Las condiciones climáticas también pueden influir en la demanda. Por ejemplo, un día caluroso podría aumentar la demanda de bebidas frías.
    
    \item Promociones y Ofertas: Las promociones especiales pueden aumentar temporalmente la demanda de ciertos productos.
\end{itemize}

\section{Tiempos Actuales en las Predicciones de Demanda}

En la actualidad, las predicciones de demanda se benefician ampliamente de las tecnologías avanzadas y el análisis de datos. Las empresas pueden aprovechar sistemas de gestión de inventario y software de análisis de datos para recopilar y procesar información histórica y en tiempo real. Estas herramientas les permiten aplicar técnicas estadísticas y modelos de pronóstico para anticipar con mayor precisión los patrones de demanda futura.

En tiempos recientes, se ha observado la aplicación de diversas técnicas en el campo de la inteligencia artificial, como sistemas expertos, y más recientemente, algoritmos genéticos. Sin embargo, a pesar de esta evolución, los modelos que han captado una atención destacada son los basados en Redes Neuronales Artificiales (RNAs). Con el transcurso del tiempo, se han desarrollado múltiples arquitecturas de RNAs para abordar una variedad de problemas en el ámbito de la predicción de demanda.


\section{Antecedentes}
En el proyecto titulado “Desarrollo de un Sistema de Control de Inventario para la Gestión de Compras de Materia Prima en el Rubro de Restaurantes” \cite{condorena2017desarrollo}. Se empleó el modelo de desarrollo de ciclo de vida en cascada para desarrollar un sistema de control de inventario para un restaurante con el objetivo de mejorar la gestión de los procesos de almacén y reducir los tiempos innecesarios en la entrega del producto final al cliente. El sistema se divide en los módulos de almacén, que ofrecen diferentes funcionalidades para el desarrollo y gestión del restaurante. Como resultado, el sistema moderniza los procesos de la empresa en el rubro de restaurantes, aportando una mejora significativa en la gestión y reducción de tiempos innecesarios en el almacén. 

\vspace{1\baselineskip}
En el trabajo titulado  “Optimización de la cadena de abastecimiento a través de un sistema inteligente de pronósticos de demanda y gestión de inventario multiproducto” \cite{pacheco2015rediseno}. Se propone una metodo de seis pasos para optimizar el área de compras y reducir costos en dos empresas del sector gastronómico: EV e IS. El objetivo general del proyecto es construir una aplicación web que semiautomatice las actividades del área de compras y optimice la cadena de abastecimiento a través de un sistema inteligente de pronósticos de demanda y gestión de inventario multiproducto. Se logró una mejora en los errores de pronóstico de 22 \%   en 2 meses, 16 \% en 8 meses y 10 \% en 12 meses. Los resultados económicos indican un VAN de \$61 millones en 3 años y una TIR de 280 \%.

\vspace{1\baselineskip}
En el artículo titulado  “Plan de gestión para la creación de una plataforma tecnológica en un establecimiento gastronómico” \cite{sanchez2018sistemas}. Los autores proponen la implementación de una plataforma tecnológica en un restaurante, combinando metodologías de gestión tradicionales y ágiles. El objetivo es mejorar la eficiencia y eficacia en las tareas de atención, supervisión, control y administración del establecimiento gastronómico. La metodología SCRUM se utilizará para la gestión del desarrollo, utilizando las estimaciones previas como métricas de evolución para identificar posibles retrasos e inconvenientes. El plan de proyecto estima los alcances, costos y duración de las tareas de desarrollo, definiendo planes de gestión de calidad y mitigación de riesgos. La combinación de metodologías utilizada busca generar una solución eficiente para la administración del desarrollo de la plataforma tecnológica. 

\vspace{1\baselineskip}
El artículo “Reducing Food Waste in the Food Industry with Deep Learning” \cite{afanador2022diseno}. Escrito por Esteban David Romero Pérez aborda el objetivo de reducir el desperdicio de alimentos en la industria alimentaria utilizando la tecnología de Deep Learning. El enfoque del artículo es ayudar a la industria alimentaria a predecir la cantidad de platos que se pueden vender en una semana específica, lo que ayudará a minimizar los desperdicios de alimentos y cumplir con el objetivo número 12 de la Organización de las Naciones Unidas para el desarrollo sostenible. Los resultados obtenidos muestran una disminución significativa en la cantidad de alimentos desperdiciados en la industria alimentaria, lo que contribuye a un futuro más sostenible.

Este capítulo usualmente es prolífico en citas de fuentes bibliográficas. Se recomienda usar el formato estándar IEEE Computer para las referencias, i.e, una lista numerada al final del artículo, ordenada alfabéticamente por el primer autor, y citada en el texto por números en corchetes \cite{ieee}. Una gran ventaja de este estilo de referenciación es que se basa en números que siempre resultan más ágiles de manipular en comparación con otros estilos que emplean combinaciones de nombres y fechas. Véanse los ejemplos de citas en este documento.

 Además, suele contener elementos tales como nombres propios, locuciones latinas y extranjeras, abreviaturas y acrónimos, símbolos gráficos de diversos significados.

Este documento auto explicado diseñado para servir de guía del informe de investigación fue elaborado en Latex (\LaTeX), el cual es un lenguaje de etiquetas de uso profesional para la divulgación del trabajo de investigación científica o tecnológica. A continuación se presentan ejemplos de elementos constitutivos de un informe de trabajo de investigación como es el TFG. Consúltese el archivo fuente \textit{tex} de este documento para ver cómo se definen tales elementos y verifíquese en este documento \textit{pdf} cómo se ve la salida obtenida en cada caso:
\begin{enumerate}
\item cómo aparece en el cuerpo del documento,
\item cómo aparece en las listas correspondientes (de acrónimos y símbolos, de figuras, de tablas y en el glosario).
\end{enumerate}
Solo se muestran casos típicos, remitiendo al lector a la copiosa ayuda que se encuentra en línea para profundizar en los detalles y dar un formato en \LaTeX\space al informe del TFG.


\textbf{Ejemplos de elementos constitutivos}
\textit{\textbf{Entradas de glosario}}

Abarca definiciones de vocablos de la jerga científica y técnica empleados en la redacción del informe del trabajo de investigación.
\begin{itemize}

\item \textit{Ejemplo No. 1.}
\newglossaryentry{electrolito}
{
name=electrolito,
description={solución capaz de conducir corriente eléctrica}
}
\begin{enumerate}
\item \textit{\Gls{electrolito}:} (mayúscula).
\item El \textit{\gls{electrolito}} (minúscula) de la pila voltaica es una solución  al 5 \% de ácido sulfúrico en agua destilada.
\item En la práctica, los \textit{\glspl{electrolito}} (plural) usualmente existen como soluciones de sales, bases o ácidos.
\end{enumerate}

\item \textit{Ejemplo No. 2.}
\newglossaryentry{linus}
{
  name=Linux,
  description={es un nombre genérico  que refiere a una familia de sistemas operativos
  			  semejantes al Unix y que usa un kernel (núcleo) común},
  first={Linu (de su creador Linus Torvald) + x:  (Linux)},
  plural={Linuces},
}

\begin{enumerate}
\item \Gls{linus} es un sistema operativo de uso libre (mayúscula en singular).
\item Existe una gran gama de distribuciones de  \Glspl{linus} (mayúscula en plural).
\item En la Facultad Politécnica se realizan muchos trabajos de investigación mediante el sistema operativo \gls{linus} (siguientes menciones).
\end{enumerate}

\item \textit{Ejemplo No. 3.}

\newglossaryentry{matri}% la etiqueta
{
name={matriz},% el vocablo
description={una tabla rectangular de elementos},% breve descripción
plural={matrices}% el plural
}
\Glspl{matri} son arreglos usualmente denotados por una letra negrita mayúscula, tal como $\mathbf{A}$. El elemento $(i,j)$-ésimo de la \gls{matri}[ A]  es usualmente denotado como $a_{ij}$. \Gls{matri}[ $\mathbf{I}$]: \gls{matri} identidad.

\end{itemize}

\textit{\textbf{Entradas de acrónimos y símbolos}}

Abarcan abreviaturas, siglas y símbolos diversos que representan conceptos y que como tales, además poseen un nombre extenso según su naturaleza. Al redactar el informe de investigación, el alumno debe recordar valerse de la gran ayuda disponible en Internet para conseguir reproducir cada objeto gráfico de manera expedita.
\begin{itemize}
\item \textit{Ejemplo No. 1. Acrónimo.}

\newacronym{svm}{SVM}{Support Vector Machine}
Primer uso: \gls{svm}\@. Siguiente uso: \gls{svm}\@. Forma corta: \acrshort{svm}\@. Forma larga: \acrlong{svm}\@. Forma completa: \acrfull{svm}\@ 
\glsreset{svm} % reinicia la bandera de primer uso

\item \textit{Ejemplo No. 2. Símbolo.}

\newacronym{PI}{\ensuremath{\pi}}{razón de la circunferencia del círculo a su diámetro}

El número \gls{PI} es una cantidad irracional, y como tal, exactamente innumerable en el sentido que no puede ser exactamente expresada en cifras: \gls{PI} = 3,141592653589793238462643383279...  Así, el valor de \gls{PI} para muchos fines prácticos suele aproximarse a 3,14.
\glsreset{PI} % reinicia la bandera de primer uso

\end{itemize}

\textit{\textbf{Símbolos y expresiones matemáticas}}


Abarca desde una simple notación o expresión en medio de un renglón hasta complejos arreglos de ecuaciones o matrices con símbolos difíciles de reproducir. Estos símbolos y expresiones requieren ser escritos en entornos matemáticos y a menudo demandan numeración secuencial que facilitan la referencia cruzada desde el texto. En \LaTeX, como en ningún otro procesador de documentos científicos, existen varios miles de símbolos matemáticos que permiten escribir prácticamente cualquier símbolo matemático que un autor pueda precisar. Esto es natural, tratándose de una herramienta informática creada a propósito para atender las necesidades de comunicación del conocimiento científico \cite{knuth, lamport}.


Las expresiones matemáticas se escriben solamente dentro de entornos matemáticos, también, en general, los símbolos propios de expresiones matemáticas. A continuación, algunos de estos entornos y expresiones matemáticas como ejemplos:

Así se escribe una ecuación en línea: $\int_{-\infty}^{\infty} e^{-x^{2}} \, dx = \sqrt{\pi}$, donde el entorno en línea está denotado por el par de apertura y cierre \$ \dots \$. Opcionalmente, se logra el mismo resultado con el par $ \backslash( \dots \backslash) $, como puede apreciarse: \( \int_{-\infty}^{\infty} e^{-x^{2}} \, dx = \sqrt{\pi} \).

Una expresión matemática desplegada en línea especial separada del texto se obtiene con el entorno matemático creado por el par de apertura y cierre $ \backslash[ \dots \backslash] $. Por ejemplo: \[ \left( \frac{1}{2} \right)^{\alpha} \] se obtiene de esta manera.

Cuando se demanda de ecuaciones numeradas, principalmente útiles para referencias cruzadas a las mismas, se emplea el siguiente entorno matemático que produce la salida correspondiente:



\begin{equation}
\sum_{i = 1}^{ \left[ \frac{n}{2} \right] }
\binom{ x_{i, i + 1}^{i^{2}} }
{ \left[ \frac{i + 3}{3} \right] }
\frac{ \sqrt{ \mu(i)^{ \frac{3}{2}} (i^{2} - 1) } }
{\sqrt[3]{\rho(i)-2} + \sqrt[3]{\rho(i) - 1}}
\end{equation}

Nótese el uso de indentación jerárquica para rastrear la estructura de la fórmula, el espaciado para resaltar las llaves y la separación de líneas para los varios pedazos de fórmulas que son más largas que una línea de texto normal. LaTeX posee la capacidad de gestión automática de numeración y contadores, de manera que el escritor no debe actualizar manualmente los cambios de número y sus respectivas referencias.


Como ejemplo final, este es un ejemplo de referencia cruzada, (teorema \ref{Pitag} y Ec. \ref{pitag}):

\begin{teorema}[Teorema de Pitágoras]
En un triángulo rectángulo, el cuadrado de la hipotenusa es igual a la suma de los cuadrados de los catetos:
\begin{equation}
hip^2 = cat_1^2 + cat_2^2
\label{pitag}
\end{equation}
\label{Pitag}
\end{teorema}
donde: \textit{hip} es la hipotenusa del triángulo rectángulo y, $ cat_1 $ y $ cat_2 $ son los catetos del mismo.
